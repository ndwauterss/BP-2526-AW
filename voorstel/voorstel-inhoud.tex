%---------- Inleiding ---------------------------------------------------------

\paragraph{Opmerking}

Dit voorstel is gebaseerd op het onderzoeksvoorstel dat werd geschreven in het kader van het vak Research Methods dat ik vorig academiejaar heb uitgewerkt.

\section{Inleiding}%
\label{sec:inleiding}

Serverless computing maakt het mogelijk voor ontwikkelaars om toepassingen te bouwen zonder het beheer van infrastructuur. Azure Functions is hierin een populair platform en wordt door Agromanager gebruikt voor functies met een dynamische workload binnen een App Service Plan. Deze service draait continu, maar wordt slechts voor korte perioden in het jaar gebruikt. Hierdoor ondervinden zij onnodige kosten en prestatieproblemen bij hoge belasting.

Door de introductie van nieuwe alternatieven die mogelijks beter aansluiten bij deze toepassingen, stelt Agromanager zich de vraag:

\textbf{In welke mate kunnen serverless container-gebaseerde oplossingen bij verschillende cloudproviders de schaalbaarheid, prestaties en kostenefficiëntie verbeteren voor Azure Functions met vaste resources en dynamische, seizoensgebonden workload?}

Om deze probleemstelling op te lossen, worden volgende deelvragen onderzocht:

\begin{enumerate}
    \item Wat zijn de beperkingen van de huidige toepassing?
    \item Wat maakt de bestaande applicatie dynamisch en seizoensgebonden? 
    \item Welke architecturale en technische factoren zijn belangrijk bij de huidige toepassing? 
    \item Aan welke eisen moet een nieuwe oplossing voldoen om als succesvol aanzien te worden?
    \item Wat zijn de belangrijkste technische verschillen tussen de mogelijke oplossingen?
    \item Wat zijn de architecturale overwegingen en uitdagingen bij een migratie naar een andere oplossing?
    \item Welke impact heeft een migratie naar de mogelijke oplossingen? 
    \item Wat is de superieure oplossing?
\end{enumerate} 

De doelstelling van dit onderzoek is om via theoretische analyse en praktisch testen een onderbouwd advies te geven aan Agromanager. Concreet zal er een Proof-of-Concept ontwikkeld worden, waarbij er een vergelijkende analyse uitgevoerd zal worden op de resultaten. Dit zal aantonen wat de impact is van een migratie naar alternatief A of B in termen van kosten, schaalbaarheid en prestaties.

%---------- Stand van zaken ---------------------------------------------------

\section{Literatuurstudie}%
\label{sec:literatuurstudie}

Hier beschrijf je de \emph{state-of-the-art} rondom je gekozen onderzoeksdomein, d.w.z.\ een inleidende, doorlopende tekst over het onderzoeksdomein van je bachelorproef. Je steunt daarbij heel sterk op de professionele \emph{vakliteratuur}, en niet zozeer op populariserende teksten voor een breed publiek. Wat is de huidige stand van zaken in dit domein, en wat zijn nog eventuele open vragen (die misschien de aanleiding waren tot je onderzoeksvraag!)?

Je mag de titel van deze sectie ook aanpassen (literatuurstudie, stand van zaken, enz.). Zijn er al gelijkaardige onderzoeken gevoerd? Wat concluderen ze? Wat is het verschil met jouw onderzoek?

Verwijs bij elke introductie van een term of bewering over het domein naar de vakliteratuur, bijvoorbeeld~\autocite{Hykes2013}! Denk zeker goed na welke werken je refereert en waarom.

Draag zorg voor correcte literatuurverwijzingen! Een bronvermelding hoort thuis \emph{binnen} de zin waar je je op die bron baseert, dus niet er buiten! Maak meteen een verwijzing als je gebruik maakt van een bron. Doe dit dus \emph{niet} aan het einde van een lange paragraaf. Baseer nooit teveel aansluitende tekst op eenzelfde bron.

Als je informatie over bronnen verzamelt in JabRef, zorg er dan voor dat alle nodige info aanwezig is om de bron terug te vinden (zoals uitvoerig besproken in de lessen Research Methods).

% Voor literatuurverwijzingen zijn er twee belangrijke commando's:
% \autocite{KEY} => (Auteur, jaartal) Gebruik dit als de naam van de auteur
%   geen onderdeel is van de zin.
% \textcite{KEY} => Auteur (jaartal)  Gebruik dit als de auteursnaam wel een
%   functie heeft in de zin (bv. ``Uit onderzoek door Doll & Hill (1954) bleek
%   ...'')

Je mag deze sectie nog verder onderverdelen in subsecties als dit de structuur van de tekst kan verduidelijken.

%---------- Methodologie ------------------------------------------------------
\section{Methodologie}%
\label{sec:methodologie}

\begin{figure*}
    \centering
    \includegraphics[width=\textwidth]{../graphics/GanttPlanningBP}
    \caption{\label{fig:gantt}Gantt diagram met de verschillende fasen en milestones van het onderzoek.}
\end{figure*}

De studie wordt op een iteratieve manier uitgevoerd, waarbij er op regelmatige basis feedback gevraagd en verwerkt wordt van de belanghebbenden. Tijdens het uitvoeren van dit onderzoek zal er ook tijd gaan naar het schrijven van een onderzoeksvoorstel en de bachelorproef. Bij de planning wordt uitgegaan van 8 uur werk per week zonder rekening te houden met eventuele vakanties, weekends en feestdagen.

\subsubsection*{Fase 1: Requirementsanalyse}

Het onderzoek start met een requirementsanalyse. Hierbij wordt een eerste analyse uitgevoerd op de toepassing van het huidige kostenmodel. Hieruit volgt een eerste requirementslijst die via een interview met de belanghebbenden wordt aangevuld om de verwachtingen en richting van het onderzoek te bepalen. De uitkomst na deze fase is een lijst van eisen die kan dienen om de literatuurstudie op te starten. In deze fase worden de volgende deelvragen beantwoord: (1) Wat zijn de beperkingen van de huidige toepassing? (2) Wat maakt de bestaande applicatie dynamisch en seizoensgebonden? (3) Welke architecturale en technische factoren zijn belangrijk bij de huidige toepassing? (4) Aan welke eisen moet een nieuwe oplossing voldoen om als succesvol aanzien te worden? De verwachte tijdsduur bij deze fase is 32 uur verspreid over 4 weken.

\subsubsection*{Fase 2: Literatuurstudie}

In de volgende stap wordt er een uitgebreide literatuurstudie uitgevoerd die alle potentiële alternatieven opsomt en waarbij rekening gehouden wordt met de eerder verkregen requirementslijst. Hierbij wordt gekeken naar container-gebaseerde oplossingen van meerdere cloud providers. Voorbeelden zijn: Azure Container Apps, Google Cloud Run en AWS Fargate. Vervolgens wordt deze longlist besproken in een opvolgmeeting met Agromanager. Hieruit komt een shortlist van één à twee alternatieven die verder vergeleken zal worden met de huidige manier van werken. Deze fase zal een antwoord geven op de deelvraag: (5) Wat zijn de belangrijkste technische verschillen tussen de mogelijke oplossingen? De tijdsinschatting voor deze fase is 80 uur verspreid over een periode van 20 weken. 

\subsubsection*{Fase 3: Proof-of-Concept}

Op dit moment kan de Proof-of-Concept (PoC) opgestart worden door het opzetten van de testomgevingen in de verschillende cloud omgevingen. Hierbij wordt een dynamische Function app gemigreerd naar de verschillende alternatieven. Bij deze migraties worden alle stappen en configuraties gedocumenteerd en zal er een antwoord komen op de deelvraag: (6) Wat zijn de architecturale overwegingen en uitdagingen bij een migratie naar een andere oplossing? Vervolgens worden relevante testscenario's uitgewerkt en uitgevoerd. De testen worden geautomatiseerd door middel van code in Python, waarbij variabelen gedefinieerd worden op één plaats. Er zal gekeken worden naar de prestaties en schaalbaarheid van elke toepassing. Dit houdt de volgende criteria in: cold start tijd, gemiddelde afhandelingstijd, throughput (aanvraag/seconde), CPU-gebruik, geheugengebruik en initialisatietijd van een nieuwe instantie.

Deze informatie wordt verkregen via Graphana en verzameld in een CSV-bestand om een verdere analyse op uit te voeren. Elke benchmark wordt minstens 30 keer uitgevoerd per testomgeving, zodat er een normaal verdeling is van de gemiddeldes (centrale limietstelling). Om de kosten te berekenen, zullen de prijzencalculators gebruikt worden van de verschillende cloud providers. Als resultaat zijn de nodige gegevens beschikbaar waarop het advies gebaseerd kan worden, alsook alle scripts die het automatiseren van de benchmarks mogelijk maken. De tijd die hiervoor nodig zal zijn, is 80 uur verspreid over 10 weken.

\subsubsection*{Fase 4: Verwerking resultaten}

Tot slot wordt er een analyse op de resultaten uitgevoerd. Dit gebeurt door het berekenen van de gemiddeldes, medianen, standaarddeviaties, minimale/maximale waarden en spreiding van resultaten. Deze nieuwe data wordt vervolgens gebruikt om statistische testen op uit te voeren. Hierbij zal gekeken worden naar een afhankelijke variabele (resultaten) en een onafhankelijke variabele (cloud provider). Deze bevindingen worden gevisualiseerd via Python en zal een antwoord bieden op de deelvragen: (7) Welke impact heeft een migratie naar de mogelijke oplossingen? (8) Wat is de superieure oplossing? Aan de hand van deze analyse en andere criteria zoals aanpassingen aan het deploymentproces en security, wordt er een advies opgesteld. Hierop kunnen de belanghebbenden zich baseren om een beslissing te nemen. Deze conclusie zal ongeveer 24 uur in beslag nemen verspreid over 3 weken.

%---------- Verwachte resultaten ----------------------------------------------
\section{Verwacht resultaat, conclusie}%
\label{sec:verwachte_resultaten}

Hier beschrijf je welke resultaten je verwacht. Als je metingen en simulaties uitvoert, kan je hier al mock-ups maken van de grafieken samen met de verwachte conclusies. Benoem zeker al je assen en de onderdelen van de grafiek die je gaat gebruiken. Dit zorgt ervoor dat je concreet weet welk soort data je moet verzamelen en hoe je die moet meten.

Wat heeft de doelgroep van je onderzoek aan het resultaat? Op welke manier zorgt jouw bachelorproef voor een meerwaarde?

Hier beschrijf je wat je verwacht uit je onderzoek, met de motivatie waarom. Het is \textbf{niet} erg indien uit je onderzoek andere resultaten en conclusies vloeien dan dat je hier beschrijft: het is dan juist interessant om te onderzoeken waarom jouw hypothesen niet overeenkomen met de resultaten.

