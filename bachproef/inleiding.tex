%%=============================================================================
%% Inleiding
%%=============================================================================

\chapter{\IfLanguageName{dutch}{Inleiding}{Introduction}}%
\label{ch:inleiding}

Serverless computing maakt het voor ontwikkelaars mogelijk om toepassingen te bouwen zonder het beheer van infrastructuur. Dit paradigma heeft echter ook nadelen. Doordat het beheer, zoals rekenkracht, uitbesteed wordt aan de providers, hebben ontwikkelaars en bijgevolg ook de IT-bedrijven minder controle over de prestaties en kosten van hun functionaliteiten. Dit kan zich vertalen in prestatieproblemen en onnodige kosten.

Tegelijkertijd dienen zich nieuwe technologieën aan die inspelen op deze pijnpunten en die ook de flexibiliteit vergroten. Hoewel deze alternatieven veelbelovend zijn, hebben veel IT-organisaties moeite met het maken van een keuze uit de steeds groter wordende opties voor hun architectuur.

\section{\IfLanguageName{dutch}{Probleemstelling}{Problem Statement}}%
\label{sec:probleemstelling}

Zo ondervindt ook Agromanager, een IT-bedrijf dat actief is in de fruitteeltsector. Zij gebruiken serverless functies die een dynamische werkbelasting hebben in een App Service Plan. Deze toepassing draait continu en wordt slechts gebruikt in korte, specifieke perioden in het jaar. Wanneer het seizoen aangebroken is dat de functies intensief gebruikt worden, ondervinden zij prestatieproblemen. Dit duidt op 2 problemen: 

\begin{enumerate}
    \item onnodige kosten door overprovisionering van bronnen
    \item prestatieproblemen door onderprovisionering van bronnen
\end{enumerate}

\section{\IfLanguageName{dutch}{Onderzoeksvraag}{Research question}}%
\label{sec:onderzoeksvraag}

Door de introductie van nieuwe alternatieven die mogelijks beter aansluiten bij deze toepassingen, stelt Agromanager zich de vraag:

\textbf{In welke mate kunnen serverless container-gebaseerde oplossingen bij verschillende cloudproviders de schaalbaarheid, prestaties en kostenefficiëntie verbeteren voor Azure Functions met dynamische en seizoensgebonden workload?}

Om deze probleemstelling op te lossen, worden volgende deelvragen onderzocht:

\begin{enumerate}
    \item Wat zijn de beperkingen van de huidige toepassing?
    \item Wat maakt de bestaande applicatie dynamisch en seizoensgebonden? 
    \item Welke architecturale en technische factoren zijn belangrijk bij de huidige toepassing? 
    \item Aan welke eisen moet een nieuwe oplossing voldoen om als succesvol aanzien te worden?
    \item Wat zijn de belangrijkste technische verschillen tussen de mogelijke oplossingen?
    \item Wat zijn de architecturale overwegingen en uitdagingen bij een migratie naar een andere oplossing?
    \item Welke impact heeft een migratie naar de mogelijke oplossingen? 
    \item Wat is de superieure oplossing?
\end{enumerate} 

\section{\IfLanguageName{dutch}{Onderzoeksdoelstelling}{Research objective}}%
\label{sec:onderzoeksdoelstelling}

De doelstelling van dit onderzoek is om via theoretische analyse en praktisch testen een onderbouwd advies te geven aan Agromanager. Concreet zal er een Proof-of-Concept ontwikkeld worden, waarbij er een vergelijkende analyse uitgevoerd zal worden op de resultaten. Dit zal aantonen wat de impact is van een migratie naar alternatief A of B in termen van kosten, schaalbaarheid en prestaties.

\section{\IfLanguageName{dutch}{Opzet van deze bachelorproef}{Structure of this bachelor thesis}}%
\label{sec:opzet-bachelorproef}

% Het is gebruikelijk aan het einde van de inleiding een overzicht te
% geven van de opbouw van de rest van de tekst. Deze sectie bevat al een aanzet
% die je kan aanvullen/aanpassen in functie van je eigen tekst.

De rest van deze bachelorproef is als volgt opgebouwd:

In Hoofdstuk~\ref{ch:stand-van-zaken} wordt een overzicht gegeven van de stand van zaken binnen het onderzoeksdomein, op basis van een literatuurstudie.

In Hoofdstuk~\ref{ch:methodologie} wordt de methodologie toegelicht en worden de gebruikte onderzoekstechnieken besproken om een antwoord te kunnen formuleren op de onderzoeksvragen.

% TODO: Vul hier aan voor je eigen hoofstukken, één of twee zinnen per hoofdstuk

In Hoofdstuk~\ref{ch:conclusie}, tenslotte, wordt de conclusie gegeven en een antwoord geformuleerd op de onderzoeksvragen. Daarbij wordt ook een aanzet gegeven voor toekomstig onderzoek binnen dit domein.