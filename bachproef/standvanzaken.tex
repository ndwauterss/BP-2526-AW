\chapter{\IfLanguageName{dutch}{Stand van zaken}{State of the art}}%
\label{ch:stand-van-zaken}

% Tip: Begin elk hoofdstuk met een paragraaf inleiding die beschrijft hoe
% dit hoofdstuk past binnen het geheel van de bachelorproef. Geef in het
% bijzonder aan wat de link is met het vorige en volgende hoofdstuk.

% Pas na deze inleidende paragraaf komt de eerste sectiehoofding.

\section{Serverless computing}

Door de opkomst van cloud computing is applicatie-ontwikkeling drastisch geëvolueerd. Dit paradigma levert geheugen, computerkracht en andere applicaties aan via het internet die op aanvraag kan op- of afschalen \autocite{Pubglob2024}. Hierdoor zijn bedrijven niet meer genoodzaakt om nieuwe software op fysieke servers of apparaten te installeren, waardoor er minder kosten zijn voor hardware en er geen kostbare tijd moet gaan naar configuraties en onderhoud \autocite{SegunFalade2024}. Ook het opschalen van hun capaciteit, wat meestal voor de aankoop van nieuwe servers en complexe integraties zorgde, is gemakkelijker en goedkoper geworden \autocite{SegunFalade2024}.

Daarnaast heeft het ook gezorgd voor nieuwe technieken om te ontwikkelen, waaronder serverless computing. Ondanks de naam worden er wel nog servers gebruikt, maar wordt het beheer van de server ontkoppelt van de ontwikkelaar \autocite{Thatikonda2023}. Deze abstractie versnelt volgens \textcite{Ghorbian2025} het proces van ontwikkelen, doordat ontwikkelaars niet meer moeten nadenken over het beheer van de onderliggende infrastructuur. De cloud providers nemen dit op zich. Dit zorgt voor betere kostenefficiëntie en schaalbaarheid \autocite{Thatikonda2023}. Anderzijds wordt er in het artikel van \textcite{Chippagiri2025} gesuggereerd dat er problemen kunnen zijn met vendor lock-in, zwakke plekken in de beveiliging, cold starts (opstart na inactiviteit) en inefficiënte resource toekenningen bij dynamische workloads, waardoor de kosten onnodig kunnen oplopen. Deze laatste 2 nadelen zijn bijzonder relevant in dit onderzoek, aangezien beiden een directe impact hebben op prestaties en kosten. 

Function-as-a-Service (FaaS), een cloud service model, is een belangrijk onderdeel van deze serverless architectuur en zorgt ervoor dat stukjes logica uitgevoerd kunnen worden als een reactie op gebeurtenissen, waaronder andere functies, databases, HTTP endpoints of een geplande actie \autocite{Yussupov2021}. Deze functies zijn stateless en short-lived, de onderliggende infrastructuur wordt beheerd door de cloud provider en de mogelijke events en third-party triggers zijn ook volledig afhankelijk van de cloud provider \autocite{Yussupov2021}.

\section{Azure Functions}

In dit onderzoek situeert het probleemdomein zich in het FaaS-gedeelte van Azure, genaamd Azure Functions. We zoomen in dit  hoofdstuk in op de mogelijkheden en beperkingen van deze technologie om zo de problemen die Agromanager ondervindt beter te begrijpen.

\subsection{Wat is het?}

Azure Functions is een serverless oplossing, waarbij de ontwikkelaar volgens \textcite{Microsoft2025} applicaties kan maken met minder code, infrastructuur en kosten. Deze functies kunnen geprogrammeerd worden in verschillende programmeertalen, waardoor je zelf de keuze hebt welke het best aansluit bij je applicatie \autocite{Microsoft2025}. 

\subsection{Use Cases}

\subsection{Technologieën}

\subsubsection{Azure WebJobs SDK}

\subsubsection{Scale Controller}

\subsubsection{Azure App Service}

\subsection{Kostenmodellen}

Omdat niet elke functie dezelfde noden heeft, zijn er verschillende hostingplannen beschikbaar waarop een Azure Function kan draaien. Deze hebben allemaal specifieke kenmerken op vlak van schaalbaarheid, kosten en prestaties.

\subsubsection{Consumption Plan}

\subsubsection{Flex Consumption Plan}

\subsubsection{Premium Plan}

\subsubsection{Dedicated Plan}

\subsubsection{Azure Container Apps}

\subsection{Prestaties \& Schaalbaarheid}

\subsection{Kostenefficiëntie}

\subsection{CI/CD}

\section{Azure Container Apps}

\subsection{Wat is het?}

\subsection{Use Cases}

\subsection{Technologieën}

\subsubsection{KEDA}

\subsubsection{Envoy}

\subsubsection{Dapr}

\subsubsection{Azure Kubernetes Service}

\subsection{Kostenmodellen}

\subsubsection{Consumption Plan}

\subsubsection{Dedicated Plan}

\subsection{Prestaties \& Schaalbaarheid}

\subsection{Kostenefficiëntie}

\subsection{CI/CD}

\section{Google Cloud Run}

\subsection{Wat is het?}

\subsection{Use Cases}

\subsection{Technologieën}

\subsubsection{Knative}

\subsubsection{gVisor}

\subsubsection{Google Kubernetes Engine}

\subsection{Kostenmodellen}

\subsubsection{Request-based}

\subsubsection{Instance-based}

\subsection{Prestaties \& Schaalbaarheid}

\subsection{Kostenefficiëntie}

\subsection{CI/CD}

\section{Vergelijkende Analyse: ACA vs GCR}